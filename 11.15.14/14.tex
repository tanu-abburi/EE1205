\documentclass[journal,12pt,twocolumn]{IEEEtran}
\usepackage{cite}
\usepackage{amsmath,amssymb,amsfonts,amsthm}
\usepackage{algorithmic}
\usepackage{graphicx}
\usepackage{textcomp}
\usepackage{xcolor}
\usepackage{txfonts}
\usepackage{listings}
\usepackage{enumitem}
\usepackage{mathtools}
\usepackage{gensymb}
\usepackage[breaklinks=true]{hyperref}
\usepackage{tkz-euclide}
\usepackage{listings}
\usepackage{circuitikz}

\DeclareMathOperator*{\Res}{Res}

\newcommand{\permcomb}[4][0mu]{{{}^{#3}\mkern#1#2_{#4}}}
\newcommand{\comb}[1][-1mu]{\permcomb[#1]{C}}

\title{Analog Assignment-1}
\author{Name -: Abburi Tanusha\\ Roll no -: EE23BTECH11201\\ Problem Assigned -: 11.15.14}

\begin{document}

\maketitle

\section*{Question:}
A wire stretched between two rigid supports vibrates in its fundamental mode with a frequency of $45 \, \text{Hz}$. The mass of the wire is $3.5 \times 10^{-2} \, \text{kg}$, and its linear mass density is $4 \times 10^{-2} \, \text{kg/m}$. The length of the wire is $0.875 \, \text{m}$. Determine the speed of a transverse wave on the string and the tension in the string.

\section*{Solution:}
Given:
\begin{table}[htb]
\centering
\caption{Input Parameters}
\begin{tabular}{|l|l|}
\hline
\textbf{Parameter} & \textbf{Value} \\
\hline
Mass per unit length of the string & $4.0 \times 10^{-2} \, \text{kg/m}$ \\
\hline
Frequency of the vibrating string & $45 \, \text{Hz}$ \\
\hline
Mass hanging from the string & $3.5 \times 10^{-2} \, \text{kg}$ \\
\hline
\end{tabular}
\end{table}
\section*{Derivation of Wave Equation:}


\textbf{1. Find the wavelength (\(\lambda\)) for the fundamental mode:}
\begin{equation}\label{eq:wavelength}
    \lambda = 2L \quad \Rightarrow \quad \lambda = 2 \times 0.875 \, \text{m} = 1.75 \, \text{m}
\end{equation}

\textbf{2. Use the frequency and wavelength to find the wave speed (\(v\)):}
\begin{equation}\label{eq:wavespeed}
    v = f \cdot \lambda \quad \Rightarrow \quad v = 45 \, \text{Hz} \times 1.75 \, \text{m} = 78.75 \, \text{m/s}
\end{equation}
\section*{Derivation of Wave Equation:}
\begin{align}\label{eq:waveequation}
    &\text{The wave equation for transverse waves on a string is derived using first principles. Starting with the tension-force relationship and applying Newton's second law, we obtain the differential equation:} \nonumber \\
    &\frac{\partial T}{\partial x} \frac{\partial^2 y}{\partial t^2} = T \frac{\partial^2 y}{\partial x^2} \mu - \frac{\partial}{\partial x}(T \frac{\partial^2 y}{\partial x^2} \mu) \nonumber \\
    &\text{After simplifying, the final wave equation is given by:} \nonumber \\
    &\frac{\partial^2 y}{\partial t^2} = \frac{T}{\mu} \frac{\partial^2 y}{\partial x^2} \nonumber \\ 
    &\text{This equation relates the wave speed (\(v\)), tension (\(T\)), and linear mass density (\(\mu\)) for transverse waves on a string:} \nonumber \\  
    &v^2 = \frac{T}{\mu}
\end{align}

\textbf{3. Use the wave speed (\(v\)) to find the tension (\(T\)) using the wave equation:}
\begin{equation}\label{eq:tension}
    T = \mu \cdot v^2 \quad \Rightarrow \quad T = (4 \times 10^{-2} \, \text{kg/m}) \times (78.75 \, \text{m/s})^2 = 123.1875 \, \text{N}
\end{equation}

Therefore, the speed of the transverse wave on the string is $78.75 \, \text{m/s}$, and the tension in the string is $123.1875 \, \text{N}$.



\end{document}

