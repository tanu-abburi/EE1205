\documentclass[journal,12pt,onecolumn]{IEEEtran}
\usepackage{cite}
\usepackage{amsmath,amssymb,amsfonts,amsthm}
\usepackage{algorithmic}
\usepackage{graphicx}
\usepackage{textcomp}
\usepackage{xcolor}
\usepackage{txfonts}
\usepackage{listings}
\usepackage{enumitem}
\usepackage{mathtools}
\usepackage{gensymb}
\usepackage[breaklinks=true]{hyperref}
\usepackage{tkz-euclide}
\usepackage{listings}
\usepackage{circuitikz}

\DeclareMathOperator*{\Res}{Res}

\newcommand{\permcomb}[4][0mu]{{{}^{#3}\mkern#1#2_{#4}}}
\newcommand{\comb}[1][-1mu]{\permcomb[#1]{C}}

\title{Analog Assignment-1}
\author{Name -: Abburi Tanusha\\ Roll no -: EE23BTECH11201\\ Problem Assigned -: 11.15.14}

\begin{document}

\maketitle

\section*{Question:}
A wire stretched between two rigid supports vibrates in its fundamental mode with a frequency of $45 \, \text{Hz}$. The mass of the wire is $3.5 \times 10^{-2} \, \text{kg}$, and its linear mass density is $4 \times 10^{-2} \, \text{kg/m}$. The length of the wire is $0.875 \, \text{m}$. Determine the speed of a transverse wave on the string and the tension in the string.

\section*{Solution:}
Given:
\begin{table}[htb]
\centering
\caption{Input Parameters}
\begin{tabular}{|l|l|}
\hline
\textbf{Parameter} & \textbf{Value} \\
\hline
Mass per unit length of the string & $4.0 \times 10^{-2} \, \text{kg/m}$ \\
\hline
Frequency of the vibrating string & $45 \, \text{Hz}$ \\
\hline
Mass hanging from the string & $3.5 \times 10^{-2} \, \text{kg}$ \\
\hline
\end{tabular}
\end{table}

To derive the expression for the speed of a transverse wave on a stretched string, let's start with Newton's second law applied to a small segment of the string. Consider a small element of the string of length $\Delta x$ and mass $\Delta m$. The tension force $T$ is acting in the positive $y$-direction, and the displacement of the string is in the $y$-direction.

The net force in the $y$-direction is given by 
\begin{equation}\label{eq:force}
    F_y = T \sin(\theta) - (T + \Delta T) \sin(\theta + \Delta \theta),
\end{equation}
where $\theta$ is the angle of displacement.

Applying Newton's second law $F = ma$ to this element in the $y$-direction:
\begin{equation}\label{eq:newton}
    \Delta m \frac{\partial^2y}{\partial t^2} = T \sin(\theta) - (T + \Delta T) \sin(\theta + \Delta \theta).
\end{equation}

For small angles $\theta$, $\sin(\theta) \approx \theta$, and we can simplify the expression:
\begin{equation}\label{eq:simplified}
    \Delta m \frac{\partial^2y}{\partial t^2} = T \theta - (T + \Delta T)(\theta + \Delta \theta).
\end{equation}

Rearrange and divide by $\Delta x$ to get the linear density $\mu$:
\begin{equation}\label{eq:linear_density}
    \frac{\partial^2y}{\partial t^2} = \frac{T}{\mu} \theta - \frac{T}{\mu} \frac{\Delta T}{T}(\theta + \Delta \theta).
\end{equation}

Now, take the limit as $\Delta x$ approaches zero and replace $\theta$ with $\frac{\partial y}{\partial x}$:
\begin{equation}\label{eq:limit}
    \frac{\partial^2y}{\partial t^2} = \frac{T}{\mu} \frac{\partial y}{\partial x} - \frac{T}{\mu} \frac{\partial}{\partial x} \left(\frac{\Delta T}{T}\frac{\partial y}{\partial x}\right).
\end{equation}

Now, assume that the displacement is a sinusoidal wave of the form $y(x, t) = A \sin(kx - \omega t)$. Substitute this into the equation and solve for $v$:
\begin{equation}\label{eq:wave_speed}
    v = \sqrt{\frac{T}{\mu}}.
\end{equation}

This completes the derivation of the wave speed using first principles.

\textbf{1. Find the wavelength (\(\lambda\)) for the fundamental mode:}
\begin{equation}\label{eq:wavelength}
    \lambda = 2L \quad \Rightarrow \quad \lambda = 2 \times 0.875 \, \text{m} = 1.75 \, \text{m}
\end{equation}

\textbf{2. Use the frequency and wavelength to find the wave speed (\(v\)):}
\begin{equation}\label{eq:wavespeed}
    v = f \cdot \lambda \quad \Rightarrow \quad v = 45 \, \text{Hz} \times 1.75 \, \text{m} = 78.75 \, \text{m/s}
\end{equation}


\textbf{3. Use the wave speed (\(v\)) to find the tension (\(T\)) using the wave equation:}
\begin{equation}\label{eq:tension}
    T = \mu \cdot v^2 \quad \Rightarrow \quad T = (4 \times 10^{-2} \, \text{kg/m}) \times (78.75 \, \text{m/s})^2 = 123.1875 \, \text{N}
\end{equation}

Therefore, the speed of the transverse wave on the string is $78.75 \, \text{m/s}$, and the tension in the string is $123.1875 \, \text{N}$.




\end{document}

