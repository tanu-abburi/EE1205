\documentclass[journal,12pt,twocolumn]{IEEEtran}
\usepackage{cite}
\usepackage{amsmath,amssymb,amsfonts,amsthm}
\usepackage{algorithmic}
\usepackage{graphicx}
\usepackage{textcomp}
\usepackage{xcolor}
\usepackage{txfonts}
\usepackage{listings}
\usepackage{enumitem}
\usepackage{mathtools}
\usepackage{gensymb}
\usepackage{comment}
\usepackage[breaklinks=true]{hyperref}
\usepackage{tkz-euclide}
\usepackage{listings}
\usepackage{gvv}
\usepackage{braket}
\def\inputGnumericTable{}
\usepackage[latin1]{inputenc}
\usepackage{color}
\usepackage{array}
\usepackage{longtable}
\usepackage{calc}
\usepackage{multirow}
\usepackage{hhline}
\usepackage{ifthen}
\usepackage{lscape}

\newtheorem{theorem}{Theorem}[section]
\newtheorem{problem}{Problem}
\newtheorem{proposition}{Proposition}[section]
\newtheorem{lemma}{Lemma}[section]
\newtheorem{corollary}[theorem]{Corollary}
\newtheorem{example}{Example}[section]
\newtheorem{definition}[problem]{Definition}
\newcommand{\BEQA}{\begin{eqnarray}}
\newcommand{\EEQA}{\end{eqnarray}}
\newcommand{\define}{\stackrel{\triangle}{=}}
\theoremstyle{remark}
\newtheorem{rem}{Remark}
\begin{document}

\bibliographystyle{IEEEtran}
\vspace{3cm}

\title{GATE 2023 CH-58}
\author{EE23BTECH11201 - Abburi Tanusha$^{*}$% <-this % stops a space
}
\maketitle
\newpage
\bigskip

\renewcommand{\thefigure}{\theenumi}
\renewcommand{\thetable}{\theenumi}

\vspace{3cm}

\maketitle
\textbf{Question:} 
A fresh catalyst is loaded into a reactor before the start of the following catalytic reaction:
\begin{align*}
  A \rightarrow \text{Products} 
\end{align*}
The catalyst gets deactivated over time. The instantaneous activity $a(t)$, at time $t$, is defined as the ratio of the rate of reaction $-r_A(t)$ (mol.$(g_{\text{cat}})^{-1}$hr$^{-1}$) to the rate of reaction with fresh catalyst. Controlled experimental measurements led to an empirical correlation:
\begin{align*}
 -r_A(t)' = -0.5t + 10
 \end{align*}
where $t$ is in hours.
The activity of the catalyst at $t=10$ hours is given by (rounded off to one decimal place):\\
\hfill(GATE 2023 CH)\\
\textbf{Solution:} 
\begin{table}[h!]
\centering
\resizebox{6cm}{!}{

\begin{tabular}{|c|c|c|}
\hline
\textbf{Values} & \textbf{Parameters} & \textbf{Description} \\
\hline
$-r_A(t)$ & $-0.25t^2 + 10t + C$ & Rate of reaction at time $t$  \\
\hline
$-r_A(0)$ & 10  & Rate of fresh catalyst  \\
\hline
$a(t)$ & $\frac{-r_A(t)}{-r_A(0)}$ & Activity of a catalyst at time $t$ \\
\hline

\end{tabular}





}
\caption{Given Parameters}
\label{tab:my_label}
\end{table}
\begin{align}
 -r_A(t)' &= -0.5t + 10 
 \end{align}
Integrating the rate expression with respect to time:
\begin{align}
 -r_A(t)' dt &= (-0.5t + 10) dt \\
 -r_A(t) &= -0.25t^2 + 10t + C 
 \end{align}
Using the initial condition t = 0 ,where $-r_A(0) = 10$ \\
\begin{align}
 -r_A(0) &= -0.25(0)^2 + 10(0) + C \\
 \implies C &=10\\
r(t) &= -0.25t^2 + 10t + 10 \\
 a(t) &= \frac{-r_A(t)}{-r_A(0)} \\
  &= \frac{-0.25t^2 + 10t + 10}{-0.25(0)^2 + 10(0) + 10}\\ 
  &= \frac{-0.25t^2 + 10t + 10}{10} 
\end{align}
Now, we calculate the activity at $t = 10$ hours:\\
\begin{align}
a(10) &= \frac{-0.25(10)^2 + 10(10) + 10}{10} \\
 &= \frac{-25 + 100 + 10}{10} \\
 &= \frac{85}{10} \\
 &= 8.5
\end{align}
\begin{figure}[h!]
\centering
\includegraphics[width=\columnwidth]{figs/gate_plot.png}
\label{fig:plot}
\caption{Activity of catalyst}
\end{figure}
\end{document}

